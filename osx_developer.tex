\documentclass[11pt]{article}

\usepackage{hyperref}
\begin{document}

\tableofcontents

\section{OSX developer install}

\subsection{Install Python}
First, you will need to download a scientific Python distribution.  We strongly recommend Anaconda.  Be warned, your computer comes with a version of Python already installed; but this pre-installed version does NOT have everything you will need to run {\bf PmagPy}, so you will still need to download Anaconda.

\begin{itemize}
   \item Download and install \href{https://www.anaconda.com/download}{Anaconda Python 3}.
   \item Open your Terminal (see \href{https://earthref.org/PmagPy/cookbook/#command_line}{this section} for more information on finding Terminal).
   \item Download a few non-default Python packages.  Run the following commands: \begin{verbatim}

    conda install future
    conda install cartopy
    pip install scripttest
    pip install --upgrade -f https://wxpython.org/Phoenix/snapshot-builds/ wxPython
\end{verbatim}
     The next two basemap packages are only required if you want to make maps:
\begin{verbatim}
    conda install basemap --channel conda-forge
    conda install basemap-data-hires --channel conda-forge

\end{verbatim}

\item If you have trouble running any of the above commands, you may need to preface them with ``sudo'': i.e., \texttt{sudo pip install scripttest}.  You will then be asked for your computer password.

  \end{itemize}



\subsection{Test your python}


To make sure that you have installed Python successfully, type \texttt{python} on your command line.  You should see something like this: \begin{verbatim}

Python 3.6.1 |Anaconda custom (x86_64)| (default, May 11 2017, 13:04:09)
[GCC 4.2.1 Compatible Apple LLVM 6.0 (clang-600.0.57)] on darwin
Type "help", "copyright", "credits" or "license" for more information.
>>>\end{verbatim}
(Press control-D to exit)


\subsection{Install PmagPy}

Here are the steps to clone and install {\bf PmagPy}.

\begin{itemize}

  \item You must use bash as your shell (not csh, zsh, etc.). This is the default for OSX, so if you have set up a different shell you will need to switch back.  Select Terminal \verb!-->! Preferences \verb!-->! General, and choose ``default login shell''.  Restart Terminal, and you'll be ready to go.

\item Download and \href{https://git-scm.com/downloads}{install git}

  \item Navigate to the directory where you want to put the PmagPy folder, then run:

\begin{verbatim}

  git clone https://github.com/PmagPy/PmagPy.git
\end{verbatim}

\item You should now have a full local copy of the PmagPy repository.  Change directories into PmagPy:

\begin{verbatim}

  cd PmagPy
\end{verbatim}

\item Next, you need to add PmagPy to your PATH so that the PmagPy programs can be called from any directory. Try running:

\begin{verbatim}

  python dev_setup.py install
\end{verbatim}

\item If you have problems with the install, run \verb!python dev_setup.py -h! for more information.  You can also set your \href{https://earthref.org/PmagPy/cookbook/#setting_path}{PATH manually} if dev\_setup.py fails.

\item After completing the developer install, you should restart your command line.

\end{itemize}

\subsection{Test PmagPy}

Test core functionality:

\begin{itemize}
  \item On your command line, type ``python'' to start the interpreter.  You will import pmagpy and then run a simple calculation.

\begin{verbatim}
>>> from pmagpy import pmag
>>> pmag.angle([350.0,10.0],[320.0,20.0])
array([30.59060998])
>>>
\end{verbatim}

\end{itemize}

Test the GUIs:

\begin{itemize}
\item  On the command line, open Pmag GUI by running:

\begin{verbatim}
pmag_gui.py
\end{verbatim}

\end{itemize}

If any of these commands don't work, go back and carefully follow the install instructions.  If you still have a problem, try the \href{https://earthref.org/PmagPy/cookbook/#trouble}{Troubleshooting section}.  If you don't find an answer there, check out the existing \href{https://github.com/PmagPy/PmagPy/issues}{Github issues} and create a new one if necessary.

\subsection{Finishing up}

Accessing example data files:

   There are many data files used in the examples of programs and for use with the textbook  \href{http://earthref.org/MAGIC/books/Tauxe/Essentials/WebBook3.html}{Essentials of Paleomagnetism}.  You can find these in PmagPy/data_files.

Installing the Etopo20 package for use with Basemap:

\begin{verbatim}
  install_etopo.py
\end{verbatim}

\subsection{Keeping PmagPy up-to-date}

    You will want to stay up to date with {\bf PmagPy} development.  To update your developer install, you will just need to navigate to the {\bf PmagPy} directory and run:

\begin{verbatim}
    git pull
\end{verbatim}

This will grab all of the latest code from \href{https://github.com/PmagPy/PmagPy}{Github}, and will be immediately available to you.

\section{Next steps}

Back to the \href{https://earthref.org/PmagPy/cookbook/#next_steps}{Cookbook}!

\end{document}
