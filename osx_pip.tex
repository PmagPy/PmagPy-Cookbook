\documentclass[11pt]{article}

\usepackage{hyperref}
\begin{document}

\tableofcontents

\section{OSX pip install}

These are the install instructions for OSX/pip.  If this isn't your operating system, or you want a developer install, you are on the wrong page.

\subsection{Install Python}

First, you will need to download a scientific Python distribution.  We strongly recommend Anaconda.  Be warned, your computer comes with a version of Python already installed; but this pre-installed version does NOT have everything you will need to run {\bf PmagPy}, so you will still need to download Anaconda.

\begin{itemize}
   \item Download and install \href{https://www.anaconda.com/download}{Anaconda Python 3}.
   \item Open your Terminal (see \href{https://earthref.org/PmagPy/cookbook/#command_line}{this section} for more information on finding Terminal).
   \item Download a few non-default Python packages.  Run the following commands: \begin{verbatim}

    conda install future cartopy
    pip install scripttest
    pip install --upgrade -f https://wxpython.org/Phoenix/snapshot-builds/ wxPython
\end{verbatim}
     The next two basemap packages are only required if you want to make maps:
\begin{verbatim}
    conda install basemap --channel conda-forge
    conda install basemap-data-hires --channel conda-forge

\end{verbatim}

\item If you have trouble running any of the above commands, you may need to preface them with ``sudo'': i.e., \texttt{sudo pip install scripttest}.  You will then be asked for your computer password.

  \end{itemize}


\subsection{Test your python}


To make sure that you have installed Python successfully, type \texttt{python} on your command line.  You should see something like this: \begin{verbatim}

Python 3.6.1 |Anaconda custom (x86_64)| (default, May 11 2017, 13:04:09)
[GCC 4.2.1 Compatible Apple LLVM 6.0 (clang-600.0.57)] on darwin
Type "help", "copyright", "credits" or "license" for more information.
>>>\end{verbatim}
(Press control-D to exit)

\subsection{Install PmagPy}

\begin{itemize}
     \item Update pip and setuptools on the command line:

\begin{verbatim}

  conda upgrade pip
  conda upgrade setuptools
\end{verbatim}
\item Install pmagpy and pmagpy-cli:

\begin{verbatim}

  pip install --upgrade pmagpy --no-deps
  pip install --upgrade pmagpy-cli --no-deps
\end{verbatim}
\item If you are getting weird install errors, try uninstalling and then force reinstalling with this:

\begin{verbatim}

  pip uninstall pmagpy &&  pip install pmagpy --upgrade --no-deps --force-reinstall --no-cache-dir
\end{verbatim}

You can do the same for pmagpy-cli.
   \end{itemize}

\subsection{Test PmagPy}

Test core functionality:

\begin{itemize}
  \item On your command line, type ``python'' to start the interpreter.  You will import pmagpy and then run a simple calculation.  Importing pmagpy may take over a minute the first time, so be patient!  It will be much faster in subsequent runs.

\begin{verbatim}
>>> from pmagpy import pmag
>>> pmag.angle([350.0,10.0],[320.0,20.0])
array([30.59060998])
>>>
\end{verbatim}

\end{itemize}

Test the GUIs:

\begin{itemize}
\item  On the command line, open Pmag GUI by running:

\begin{verbatim}
pmag_gui_anaconda
\end{verbatim}

\end{itemize}

If any of these commands don't work, go back and carefully follow the install instructions.  If you still have a problem, try the \href{https://earthref.org/PmagPy/cookbook/#trouble}{Troubleshooting section}.  If you don't find an answer there, check out the existing \href{https://github.com/PmagPy/PmagPy/issues}{Github issues} and create a new one if necessary.

%\customlink{finishing_up}
\subsection{Finishing up}

Accessing example data files:

   There are many data files used in the examples of programs and for use with the textbook  \href{http://earthref.org/MAGIC/books/Tauxe/Essentials/WebBook3.html}{Essentials of Paleomagnetism}.     You may want to copy  the data files to your Desktop or another convenient location.
   To do this, navigate on the command line to your desired destination folder (for help, see \href{https://earthref.org/PmagPy/cookbook/#file_system}{this section}).  Then, use the command:

\begin{verbatim}
  move_data_files.py
\end{verbatim}

This will copy all of the PmagPy example files to your current directory.  NB: If you have a developer install, you can simply navigate to PmagPy/data\_files, and move\_data\_files.py will not be needed.

%\customlink{hires}
Installing the Etopo20 package for use with Basemap:

\begin{verbatim}
  install_etopo.py
\end{verbatim}



\subsection{Keeping PmagPy up-to-date}

To stay up to date with new features and bug fixes, you should periodically update both {\bf PmagPy} packages.
\begin{verbatim}

  pip install pmagpy --upgrade --no-deps
\end{verbatim}

To check the currently installed version number for pmagpy (or any other Python package), run:
\begin{verbatim}
conda list
\end{verbatim}

If you ever need to uninstall pmagpy or pmagpy-cli:

\begin{verbatim}

  pip uninstall pmagpy
\end{verbatim}
  or
\begin{verbatim}
  pip uninstall pmagpy-cli
\end{verbatim}

\section{Next steps}

Back to the \href{https://earthref.org/PmagPy/cookbook/#next_steps}{Cookbook}!


\end{document}
