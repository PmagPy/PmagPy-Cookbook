\documentclass[11pt]{article}

\usepackage{hyperref}
\begin{document}

\tableofcontents


\section{Linux pip install}

These are the install instructions for Linux/pip.  If this isn't your operating system, or you want a developer install, go back to the \href{https://earthref.org/PmagPy/cookbook/#next_steps}{Cookbook} and select the correct link.


\subsection{Install Python}

First, you will need to download a scientific Python distribution.  We strongly recommend Anaconda.  Be warned, your computer comes with a version of Python already installed; but this pre-installed version does NOT have everything you will need to run {\bf PmagPy}, so you will still need to download Anaconda.

\begin{itemize}
\item Download and install \href{https://www.anaconda.com/download}{Anaconda Python 3}.
  \item In general, if you run into errors with your Anaconda Python environment, it can save time to simply \href{https://docs.anaconda.com/anaconda/install/uninstall/}{uninstall and reinstall}.
   \item Open your command line (Terminal).  See \href{https://earthref.org/PmagPy/cookbook/#command_line}{this Cookbook section} for more information on finding your command line.
   \item Create and activate a new conda Python environment with some required packages: \begin{verbatim}

    conda create -n pmagpy_env future wxPython cartopy pandas matplotlib requests
    conda activate pmagpy_env
    pip install --user --upgrade pip setuptools
    conda install conda-forge::python-wget
\end{verbatim}
   \item You have now created a new environment called ``pmagpy\_env'' and activated it.  This gives you a Python environment with the packages you need to run PmagPy programs.  Each time you want to use PmagPy, you will open your Terminal and then run:

\begin{verbatim}

     conda activate pmagpy_env
\end{verbatim}
     You can also deactivate the environment using:

     \begin{verbatim}
       conda deactivate
       \end{verbatim}

     You must always activate your pmagpy environment when you open a new Command Prompt window if you want to use PmagPy in that window.  To learn more about managing conda environments, etc., see the Anaconda documentation and \href{https://know.continuum.io/rs/387-XNW-688/images/conda-cheatsheet.pdf}{cheatsheet}.
   \item When running these commands you may see some warnings about dependencies, but you can just ignore them!

\item If a package fails to install, you may need to use ``sudo'': i.e., \texttt{sudo pip install scripttest}.  You will then be asked for your computer password.

  \end{itemize}


\subsection{Install wxPython}

Next, you need to install wxPython.

Most recently (on Ubuntu 18.04.2), I've had success with simply running:

\begin{verbatim}
    conda install wxPython
\end{verbatim}

However, in the past this didn't work and still may not work for all Linux distributions.  Try the above command first, and if that fails, try working through the instructions below.

\begin{itemize}
  \item The following is a list of wxPython dependencies for Ubuntu 16.04 (this list may be redundant or incomplete, depending on your specific OS):

  \begin{verbatim}

    pip install libtiff

    sudo apt-get install libgtk-3-dev freeglut3-dev libgstreamer1.0-0 gstreamer1.0-plugins-base gstreamer1.0-plugins-good gstreamer1.0-plugins-bad gstreamer1.0-plugins-ugly gstreamer1.0-libav gstreamer1.0-doc gstreamer1.0-tools libgstreamer-plugins-base0.10-dev

    sudo apt-get install dpkg-dev build-essential libjpeg-dev  libtiff-dev libsdl1.2-dev   libnotify-dev  freeglut3 freeglut3-dev libsm-dev libwebkitgtk-3.0-dev libxtst-dev python-dev libpng-dev

    sudo apt-get install libwebkit2gtk-4.0-dev libsdl2-dev
\end{verbatim}


Carefully read through \href{https://wxpython.org/pages/downloads/}{this page} and look for a compatible wheel for your distribution.  Download the appropriate wheel and then install it:
\begin{verbatim}

pip install wxPython-4.0.1-cp35-cp35m-linux_x86_64.whl

\end{verbatim}

Be sure to replace the wheel name above with the actual wheel you downloaded.

Or install the wheel from the download site directly:
\begin{verbatim}
pip install -U -f https://extras.wxpython.org/wxPython4/extras/linux/gtk3/ubuntu-16.04 wxPython

\end{verbatim}

Again, you will need the replace the link above with the link appropriate for your OS.

\item If you aren't able to find a wheel that works, try \href{https://wxpython.org/blog/2017-08-17-builds-for-linux-with-pip/}{these instructions}

\end{itemize}

\subsection{Test your python}


Run python on your command line.

You should see something like this: \begin{verbatim}

Python 3.6.1 |Anaconda custom (x86_64)| (default, May 11 2017, 13:04:09)
[GCC 4.2.1 Compatible Apple LLVM 6.0 (clang-600.0.57)] on darwin
Type "help", "copyright", "credits" or "license" for more information.
>>>\end{verbatim}

If you don't see that, go back and reinstall Anaconda.

Once you have successfully started the python interactive prompt, try:
\begin{verbatim}

    import wx

\end{verbatim}
  You may have an error like this:
\begin{verbatim}
    ImportError: libwx_gtk3u_core-3.0.so.0: cannot open shared object file: No such file or directory

\end{verbatim}

If so, exit the python interpreter (quit()) and try:
  \begin{verbatim}
    export LD_LIBRARY_PATH=~/anaconda3/lib/python3.6/site-packages/wx/

\end{verbatim}

(If you are using a non-Anaconda Python, your path to wx may be different).

Try again to import wx.  If it now works, you may want to add that line to the end of your .bashrc file so that you don't have to specify the path each time.

For more details about this issue, see \href{https://github.com/wxWidgets/Phoenix/blob/e13273c5d939d993abf2a2649e90b3ea0d39382c/packaging/README-bdist.txt#L38-L57}{this explanation} and \href{https://github.com/pyenv/pyenv/issues/691}{this Github issue} for more information.

\subsection{Install PmagPy}

\begin{itemize}
\item Install pmagpy and pmagpy-cli:

\begin{verbatim}

  pip install --upgrade pmagpy --no-deps
  pip install --upgrade pmagpy-cli --no-deps
\end{verbatim}
     \item If you are getting weird install errors, try uninstalling and then force reinstalling with this:

\begin{verbatim}

  pip uninstall pmagpy pmagpy-cli
  pip install pmagpy --upgrade --no-deps --force-reinstall --no-cache-dir
  pip install pmagpy-cli --upgrade --no-deps --force-reinstall --no-cache-dir
\end{verbatim}

   \end{itemize}


\subsection{Test PmagPy}

Test core functionality:

\begin{itemize}
  \item On your command line, type ``python'' to start the interpreter.  You will import pmagpy and then run a simple calculation.  Importing pmagpy may take over a minute the first time, so be patient!  It will be much faster in subsequent runs.

\begin{verbatim}
>>> from pmagpy import pmag
>>> pmag.angle([350.0,10.0],[320.0,20.0])
array([30.59060998])
>>>
\end{verbatim}

\end{itemize}

Test the GUIs:

\begin{itemize}
\item  On the command line, open Pmag GUI by running:

\begin{verbatim}
pmag_gui.py
\end{verbatim}

\end{itemize}

If you get an error about ``pythonw'', you may need to link ``python'' to ``pythonw'' with a command like:

  \begin{verbatim}

    ln -s ~/anaconda3/bin/python3 ~/anaconda3/bin/pythonw

    \end{verbatim}

If you get any other errors, or if any of the test commands don't work, go back and carefully follow the install instructions to make sure you didn't miss a step.  If you still have a problem, try the \href{https://earthref.org/PmagPy/cookbook/#trouble}{Troubleshooting section}.  If you don't find an answer there, check out the existing \href{https://github.com/PmagPy/PmagPy/issues}{Github issues} and create a new one if necessary.


\subsection{Finishing up}

Accessing example data files:

   There are many data files used in the examples of programs and for use with the textbook  \href{http://earthref.org/MAGIC/books/Tauxe/Essentials/WebBook3.html}{Essentials of Paleomagnetism}.     You may want to copy  the data files to your Desktop or another convenient location.
   To do this, navigate on the command line to your desired destination folder (for help, see \href{https://earthref.org/PmagPy/cookbook/#file_system}{this Cookbook section}).  Then, use the command:

\begin{verbatim}
  move_data_files.py
\end{verbatim}

This will copy all of the PmagPy example files to your current directory.  NB: If you have a developer install, you can simply navigate to PmagPy/data\_files, and move\_data\_files.py will not be needed.


\subsection{Keeping PmagPy up-to-date}


To stay up to date with new features and bug fixes, you should periodically update both {\bf PmagPy} packages.
\begin{verbatim}

  pip install pmagpy --upgrade --no-deps
\end{verbatim}

To check the currently installed version number for pmagpy (or any other Python package), run:
\begin{verbatim}
  conda list
\end{verbatim}

If you ever need to uninstall pmagpy or pmagpy-cli:

\begin{verbatim}

  pip uninstall pmagpy
\end{verbatim}
  or
\begin{verbatim}
  pip uninstall pmagpy-cli
\end{verbatim}

\section{Next steps}

Back to the \href{https://earthref.org/PmagPy/cookbook/#next_steps}{Cookbook}!


\end{document}
