\documentclass[11pt]{article}

\usepackage{hyperref}
\begin{document}

\tableofcontents

\section{Linux developer install}

\subsection{Install Python}

First, you will need to download a scientific Python distribution.  We strongly recommend Anaconda.  Be warned, your computer comes with a version of Python already installed; but this pre-installed version does NOT have everything you will need to run {\bf PmagPy}, so you will still need to download Anaconda.

\begin{itemize}
   \item Download and install \href{https://www.anaconda.com/download}{Anaconda Python 3}.
   \item Open your Terminal (see \href{https://earthref.org/PmagPy/cookbook/#command_line}{this section} for more information on finding Terminal).
   \item Download a few non-default Python packages.  Run the following commands: \begin{verbatim}

    conda install future
    conda install cartopy
    pip install scripttest
    pip install --upgrade -f https://wxpython.org/Phoenix/snapshot-builds/ wxPython
\end{verbatim}
     The next two basemap packages are only required if you want to make maps:
\begin{verbatim}
    conda install basemap --channel conda-forge
    conda install basemap-data-hires --channel conda-forge

\end{verbatim}

\item If you have trouble running any of the above commands, you may need to preface them with ``sudo'': i.e., \texttt{sudo pip install scripttest}.  You will then be asked for your computer password.

  \end{itemize}

\subsection{Install wxPython}

Next, you need to install wxPython.

\begin{itemize}
  \item The following is a list of wxPython dependencies for Ubuntu 16.04 (this list may be redundant or incomplete, depending on your specific OS):

  \begin{verbatim}

    pip install libtiff

    sudo apt-get install libgtk-3-dev freeglut3-dev libgstreamer1.0-0 gstreamer1.0-plugins-base gstreamer1.0-plugins-good gstreamer1.0-plugins-bad gstreamer1.0-plugins-ugly gstreamer1.0-libav gstreamer1.0-doc gstreamer1.0-tools libgstreamer-plugins-base0.10-dev

    sudo apt-get install dpkg-dev build-essential libjpeg-dev  libtiff-dev libsdl1.2-dev   libnotify-dev  freeglut3 freeglut3-dev libsm-dev libwebkitgtk-3.0-dev libxtst-dev python-dev libpng-dev

    sudo apt-get install libwebkit2gtk-4.0-dev libsdl2-dev
\end{verbatim}


Carefully read through \href{https://wxpython.org/pages/downloads/}{this page} and look for a compatible wheel for your distribution.  Download the appropriate wheel and then install it:
\begin{verbatim}

pip install wxPython-4.0.1-cp35-cp35m-linux_x86_64.whl

\end{verbatim}

Be sure to replace the wheel name above with the actual wheel you downloaded.

Or install the wheel from the download site directly:
\begin{verbatim}
pip install -U -f https://extras.wxpython.org/wxPython4/extras/linux/gtk3/ubuntu-16.04 wxPython

\end{verbatim}

Again, you will need the replace the link above with the link appropriate for your OS.

\item If you aren't able to find a wheel that works, try \href{https://wxpython.org/blog/2017-08-17-builds-for-linux-with-pip/}{these instructions}

\end{itemize}


\subsection{Test your python}

Run python on your command line.

You should see something like this: \begin{verbatim}

Python 3.6.1 |Anaconda custom (x86_64)| (default, May 11 2017, 13:04:09)
[GCC 4.2.1 Compatible Apple LLVM 6.0 (clang-600.0.57)] on darwin
Type "help", "copyright", "credits" or "license" for more information.
>>>\end{verbatim}

If you don't see that, go back and reinstall Anaconda.

Once you have successfully started the python interactive prompt, try:
\begin{verbatim}

    import wx

\end{verbatim}
You may have an error like this:
\begin{verbatim}
    ImportError: libwx_gtk3u_core-3.0.so.0: cannot open shared object file: No such file or directory

\end{verbatim}

If so, exit the python interpreter (quit()) and try:
  \begin{verbatim}
    export LD_LIBRARY_PATH=~/anaconda3/lib/python3.6/site-packages/wx/

\end{verbatim}

(If you are using a non-Anaconda Python, your path to wx may be different).

Try again to import wx.  If it now works, you may want to add that line to the end of your .bashrc file so that you don't have to specify the path each time.

For more details about this issue, see \href{https://github.com/wxWidgets/Phoenix/blob/e13273c5d939d993abf2a2649e90b3ea0d39382c/packaging/README-bdist.txt#L38-L57}{this explanation} and \href{https://github.com/pyenv/pyenv/issues/691}{this Github issue} for more information.


\subsection{Install PmagPy}



Here are the steps to clone and install {\bf PmagPy}.

\begin{itemize}

\item You must use bash as your shell (not csh, zsh, etc.).  You can make sure you are using bash with the command:
\begin{verbatim}

  printenv SHELL
\end{verbatim}

If your result is not ``/bin/bash'', you will need to reset it.   See \href{https://stackoverflow.com/questions/13046192/changing-default-shell-in-linux}{this question} for more details.

\item Download and \href{https://git-scm.com/downloads}{install git}

  \item Navigate to the directory where you want to put the PmagPy folder, then run:

\begin{verbatim}

  git clone https://github.com/PmagPy/PmagPy.git
\end{verbatim}

\item You should now have a full local copy of the PmagPy repository.  Change directories into PmagPy:

\begin{verbatim}

  cd PmagPy
\end{verbatim}

\item Next, you need to add PmagPy to your PATH so that the PmagPy programs can be called from any directory. Try running:

\begin{verbatim}

  python dev_setup.py install
\end{verbatim}

\item If you have problems with the install, run \verb!python dev_setup.py -h! for more information.  You can also set your \href{https://earthref.org/PmagPy/cookbook/#setting_path}{PATH manually} if dev\_setup.py fails.

\item After completing the developer install, you should restart your command line.

\end{itemize}

\subsection{Test PmagPy}


Test core functionality:

\begin{itemize}
  \item On your command line, type ``python'' to start the interpreter.  You will import pmagpy and then run a simple calculation.

\begin{verbatim}
>>> from pmagpy import pmag
>>> pmag.angle([350.0,10.0],[320.0,20.0])
array([30.59060998])
>>>
\end{verbatim}

\end{itemize}

Test the GUIs:

\begin{itemize}
\item  On the command line, open Pmag GUI by running:

\begin{verbatim}
pmag_gui.py
\end{verbatim}

\end{itemize}

If any of these commands don't work, go back and carefully follow the install instructions.  If you still have a problem, try the \href{https://earthref.org/PmagPy/cookbook/#trouble}{Troubleshooting section}.  If you don't find an answer there, check out the existing \href{https://github.com/PmagPy/PmagPy/issues}{Github issues} and create a new one if necessary.

\subsection{Finishing up}

Accessing example data files:

   There are many data files used in the examples of programs and for use with the textbook  \href{http://earthref.org/MAGIC/books/Tauxe/Essentials/WebBook3.html}{Essentials of Paleomagnetism}.  You can find these in PmagPy/data_files.


Installing the Etopo20 package for use with Basemap:

\begin{verbatim}
  install_etopo.py
\end{verbatim}

\subsection{Keeping PmagPy up-to-date}


    You will want to stay up to date with {\bf PmagPy} development.  To update your developer install, you will just need to navigate to the {\bf PmagPy} directory and run:

\begin{verbatim}
    git pull
\end{verbatim}

This will grab all of the latest code from \href{https://github.com/PmagPy/PmagPy}{Github}, and will be immediately available to you.

\section{Next steps}

Back to the \href{https://earthref.org/PmagPy/cookbook/#next_steps}{Cookbook}!

\end{document}
