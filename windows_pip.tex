\documentclass[11pt]{article}

\usepackage{hyperref}
\begin{document}

\tableofcontents

\section{Windows pip install}

These are the install instructions for Windows/pip.  If this isn't your operating system, or you want a developer install, go back to the \href{https://earthref.org/PmagPy/cookbook/#next_steps}{Cookbook} and select the correct link.


\subsection{Install Python}
First, you will need to download a scientific Python distribution.  We strongly recommend Anaconda.

   \begin{itemize}
   \item Download and install \href{https://www.anaconda.com/download}{Anaconda Python 3}.  Select an install for “Just Me”, not for all users.  You should select ``Add Anaconda to my PATH environment variable'' in Advanced Installation Options (if available).  Otherwise, stick with the defaults.
   \item Open your command line (Command Prompt).  See \href{https://earthref.org/PmagPy/cookbook/#command_line}{this Cookbook section} for more information on finding Command Prompt)
   \item Note: if you did not add Anaconda Python to your PATH as recommended above, you will open the ``Anaconda Prompt'' instead).
   \item Create and activate a new conda Python environment with some required packages: \begin{verbatim}

    conda create -n pmagpy_env future wxPython cartopy pandas matplotlib requests jupyter
    conda activate pmagpy_env
    pip install --user --upgrade pip setuptools
    conda install conda-forge::python-wget
\end{verbatim}
   \item You have now created a new environment called ``pmagpy\_env'' and activated it.  This gives you a Python environment with the packages you need to run PmagPy programs.  Each time you want to use PmagPy, you will open your Command Prompt and then run:

     conda activate pmagpy\_env

     You can also deactivate the environment using:

     conda deactivate

     You must always activate your pmagpy environment when you open a new Command Prompt window if you want to use PmagPy in that window.  To learn more about managing conda environments, etc., see the Anaconda documentation and \href{https://know.continuum.io/rs/387-XNW-688/images/conda-cheatsheet.pdf}{cheatsheet}.

\end{itemize}


\subsection{Test your python}

To make sure that you have installed Python successfully, type \texttt{python} on your command line.  You should see something like this: \begin{verbatim}

Python 3.6.4 |Anaconda custom (32-bit)| (default, Jan 16 2018, 10:21:59) [MSC v.1900 32 bit (Intel)] on win32
Type "help", "copyright", "credits" or "license" for more information.
>>>\end{verbatim}
(Type ``quit()'' and then hit enter to end your python session)


\subsection{Install PmagPy}


\begin{itemize}
\item Install pmagpy and pmagpy-cli into your pmagpy_env:

\begin{verbatim}

  pip install --upgrade pmagpy --no-deps
  pip install --upgrade pmagpy-cli --no-deps
\end{verbatim}
     \item If you are getting weird install errors, try uninstalling and then force reinstalling with this:

\begin{verbatim}

  pip uninstall pmagpy pmagpy-cli
  pip install pmagpy --upgrade --no-deps --force-reinstall --no-cache-dir
  pip install pmagpy-cli --upgrade --no-deps --force-reinstall --no-cache-dir
\end{verbatim}

   \end{itemize}

\subsection{Test PmagPy}

Test core functionality:

\begin{itemize}
  \item On your command line, type ``python'' to start the interpreter.  You will import pmagpy and then run a simple calculation.  Importing pmagpy may take over a minute the first time, so be patient!  It will be much faster in subsequent runs.

\begin{verbatim}
>>> from pmagpy import pmag
>>> pmag.angle([350.0,10.0],[320.0,20.0])
array([30.59060998])
>>>
\end{verbatim}

\end{itemize}

Test the GUIs:

\begin{itemize}
\item  On the command line, open Pmag GUI by running:

\begin{verbatim}
pmag_gui
\end{verbatim}

\end{itemize}

Remember that the program may be very slow to initialize the first time!  You may also need to resize the GUI window.  If the program doesn't open or you get an error message, go back and carefully follow the install instructions to make sure you didn't miss a step.  If you still have a problem, try the \href{https://earthref.org/PmagPy/cookbook/#trouble}{Troubleshooting section}.  If you don't find an answer there, check out the existing \href{https://github.com/PmagPy/PmagPy/issues}{Github issues} and create a new one if necessary.

\subsection{Finishing up}

Accessing example data files:

   There are many data files used in the examples of programs and for use with the textbook  \href{http://earthref.org/MAGIC/books/Tauxe/Essentials/WebBook3.html}{Essentials of Paleomagnetism}.     You may want to copy  the data files to your Desktop or another convenient location.
   To do this, navigate on the command line to your desired destination folder (for help, see \href{https://earthref.org/PmagPy/cookbook/#file_system}{this Cookbook section}).  Then, use the command:

\begin{verbatim}
  move_data_files
\end{verbatim}

This will copy all of the PmagPy example files to your current directory.  NB: If you have a developer install, you can simply navigate to PmagPy/data\_files, and move\_data\_files will not be needed.


\subsection{Keeping PmagPy up-to-date}


To stay up to date with new features and bug fixes, you should periodically update both {\bf PmagPy} packages.
\begin{verbatim}

  pip install pmagpy --upgrade --no-deps
\end{verbatim}

To check the currently installed version number for pmagpy (or any other Python package), run:
\begin{verbatim}
  conda list
\end{verbatim}

If you ever need to uninstall pmagpy or pmagpy-cli:

\begin{verbatim}

  pip uninstall pmagpy
\end{verbatim}
  or
\begin{verbatim}
  pip uninstall pmagpy-cli
\end{verbatim}

\section{Next steps}

Back to the \href{https://earthref.org/PmagPy/cookbook/#next_steps}{Cookbook}!


\end{document}
